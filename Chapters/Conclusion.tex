\newpage
\chapter{Conclusion}
\label{Conclusion}

Population growth and increasing urbanisation lead to more and more congestion problems in cities. This is not only annoying, but also has large environmental and economic impact. The challenge for this project is to create a sustainable transport network that will reduce the congestion problems, but is also economically feasible. For this reason an Urban Air Mobility System is designed and implementation into the existing transport network is investigated. 

In this report the commuting flows of Los Angeles have been investigated and visualised to find the most populated and congested routes. These routes will be the main target of the system design, as the most time profit can be obtained here, which results in a higher feasibility of the system succeeding. To find what kind of vehicle and network is optimal for servicing these routes, a tool is created to find values of the quantitative criteria. Multiple concepts have been designed, with different passenger capacities. Three ranges have been observed: 1-2, 4-6 and 20+ passengers. The winning concepts in each passenger range will be compared in a trade-off and during the Mid Term Review the best concept will be chosen. 

Following this Mid Term Report, the winning concept will be worked out in more detail, regarding the vehicle, infrastructure and operations. The network of the air mobility system will be described for the city of Los Angeles and the flexibility of the system is investigated by implementing the system in other cities. 