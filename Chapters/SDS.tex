\newpage
\chapter{Sustainable Development Strategy}
\label{ch-SDS}
To make sure sustainability is considered throughout the complete design process, in every report a description on the approach to sustainable design will be provided. For each major step taken in the process leading into the Midterm Report, an explanation of sustainability is taken into account. A division is made between the three forms of sustainability, namely environmental, social and economic sustainability. In \autoref{definition} these definitions are explained, followed by the phases of the project where sustainability is most taken into account. In \autoref{sustcrit} the sustainability criteria for the trade-off are discussed, after which the trade-off will be explained in \autoref{tradeoff}




\section{Definition}
\label{definition}
Environmental sustainability is the best known form of sustainability. It describes all aspects of the system which have an impact on the environment, like pollution, usage of scarce materials and energy use during manufacturing and operations.

To enable the development of the system, it has to be economically feasible, but better yet economically sustainable. Various considerations have to be taken into account in order to minimise the development, manufacturing and operation costs, but also the ticket prices should be affordable.

Social sustainability describes whether a product or system adds something to the quality of life. This could be interpreted in several ways; a product or system could provide a way to improve the health of the society, or it could improve access to other places by providing a transport service.

Although the three different forms of sustainability are now described separately, often the definitions overlap and interact with each other. 


\section{Criteria}
\label{sustcrit}
The trade-off is one of the most crucial phases for taking sustainability into account, as this is the stage where a lot of decisions will be made on what the system will come to look like. In \autoref{CriteriaTools}, the criteria for this trade-off will be defined and the tools to determine this are explained. The main criteria are divided into multiple sub-criteria, in which different forms of sustainability are represented. The main criteria that have a sustainability aspect are Ecological Sustainability, Social Acceptance and Cost/Profit. 

The most obvious criterion regarding sustainability is the Ecological Sustainability, consisting of the energy used per passenger kilometre and the battery mass. The latter is mostly focused on the scarce materials often used in batteries. A high battery mass points at more use of these rare materials that are very hard to recycle and have a high environmental impact. Hence, the total battery mass of the system should be minimised.

Another sub-criterion in which sustainability is embedded is that of the energy used per passenger kilometre. This will define for a large part the environmental sustainability, as this is a measure for the overall energy efficiency of the system. The greater the energy used per passenger kilometre for a given system, means more energy is required to achieve a certain transport goal. This makes the system less efficient. Energy efficiency evaluates the level of environmental sustainability because it is a direct indicator of the efficiency in consuming resources to produce energy. 35\% of US electricity generation is powered via the consumption of non-renewable natural gas in 2018, making it the largest source of energy generation\footnote{\url{https://www.eia.gov/energyexplained/index.php?page=electricity_in_the_united_states}[accessed 17.05.19]}. It is therefore key to maintain a high level of energy efficiency for the system. Other criteria such disposition of system waste, and whether vehicle materials are renewable cannot be evaluated at this stage of design, as they require more information on the materials being used in the system and infrastructure. 

Social acceptance depends on the the noise and downwash that the vehicle produces and will be the main drivers for that criteria. If the noise and downwash levels are high, the system will likely not be a long term solution to the mobility problems that are faced in 2050. To ensure that the quality of life is indeed increased with the introduction of a UAM system, it is crucial that this aspect of social sustainability is taken into consideration and that the 'social acceptance' of a solution has a relatively high weight in the trade-off. 
%Another important aspect is the direct user experience, which includes the level of comfort and safety being experienced. Several of these factors are however user subjective and need further analysis of a specific chosen system design and therefore will be further investigated in later stages of the design process.
%There are other important aspects to social acceptance that are not directly evaluated such as whether the surrounding communities appreciate the presence of air mobility vehicles in the sky and urban landscapes, whether such a mode of transport is culturally accepted. 

Also the passenger experience and safety are part of the social sustainability. When the mobility system is not comfortable, easy to use or has a high failure rate, commuters are unlikely to use the system. If this would be the case, no improvement in the quality of life can be obtained and the UAM system would be socially unsustainable. Therefore, the comfort and the experience of the vehicle and vertiport has to be taken into account. Also, vehicle with a lot of moving rotors or wings have a higher chance of failure, which would lead to a less safe vehicle. This is also looked at in the creation of the risk map of the final concept.

The ticket price is the principal measure of the economic sustainability of the system. It is difficult to predict how costs and demand for the final system will behave over time, however it is sufficient at this stage to evaluate the total development, production and operation costs, and average ticket cost per trip. This will give a basis of comparison of the cost required to develop and maintain the different systems, as well as how economically feasible it is for the customer to access the mode of transport. 

Determining the weights of the different criteria is done by everyone in the team individually, to get balanced weighting factors for the trade-off. By not keeping the sustainability one big criterion but spreading it out over multiple smaller criteria, it is made sure that the sustainability is taken into consideration in the different parts of the design process.

\section{Future design plan}
The next phase of this Design Synthesis Exercise, will mostly be focused on designing the system and vehicle in more detail. In this phase still sustainability has to be considered in every choice that is made, to make sure the mobility system will come out to be a sustainable competition to existing polluting transport modes.  In the final trade-off, the weighted criteria described in \autoref{sustcrit} are used to find the 'winning' concept. Because sustainability has been taken into account during the determination of the criteria, the 'winning' concept will cover the different forms of sustainability.


