\newpage
\chapter{Executive Summary}
\label{ch-ES}

Due to the increasing urbanisation, the pressure on road transportation and public transportation networks will keep on growing in the coming years. To solve this problem and deal with the growing energy demands, an urban air mobility system for the year 2050 will be developed in this project. The system will be designed for transporting passengers and not cargo, as the team wants to solve congestion. The report contains an extended market analysis that elaborates on the congestion and user needs in the city taken as a reference, Los Angeles. A sustainable development strategy follows, in which the approach to sustainability throughout the project is explained. Next, the method to trade-off the different solution concepts is discussed, which is followed by an overview of the criteria and tools used in this trade-off process. Two trade-off processes will be performed. First to determine the most feasible option in three different classes of vehicle sizes, namely 1-2 pax, 4-6 pax and 20+ pax, the next to determine which of these three options is best. The concept is sized for batteries because this is easier than hydrogen and there is more data available. The final choice of concept is postponed to after the midterm deadline to take into consideration the knowledge and expertise of the tutors and coaches. 
%MENTION SOMEWHERE WHY WE HAVE CHOSEN NOT TO TRANSPORT CARGO (DOES NOT SOLVE CONGESTION) AND WHY WE HAVE SIZED THE CONCEPT FOR BATTERIES (MUCH EASIER THAN HYDROGEN, DATA AVIALABLE. COULD HAVE INCLUDED SUPERCAPACITOR CONCEPTS).

\section{Extended Market Analysis}
The goal of the extended market analysis is to further investigate how a UAM system can be best implemented in the case study for Los Angeles. Possible ways to decrease congestion level have been studied in the past and amongst others include the implementation of additional taxes on driving and the improvement of the current public transport system. During the development of a UAM system, these studies and the effect of the system on the existing transport flows must be taken into account. The average user in LA needs a transport system that is most importantly time efficient, affordable and sustainable in the long run. 

Data from the Census Transportation Planning Products (CTPP) was used to create a mapping of the traffic flows in LA at different times of the day. It includes amongst other things the trip time, trip distance and the density of people moving where. In the database which was used, LA is divided into Public Use Microdata Areas (PUMAs) and the transport by car between each respective PUMA is computed and plotted on a map. This data was used to identify the most interesting routes for a UAM system in LA. 

Even when designing a UAM system that runs efficiently, it is interesting to look at the broader picture and keep in mind that other new forms of transport might be present in 2050. A brief discussion about a few possible options takes place, along with a first estimate of the energy used per passenger per kilometre for each of the modes of transport. The latter is included to get a first comparison between the UAM system and other forms of transport in terms of energy efficiency. 

\section{Sustainable Development Strategy}
A sustainable development strategy is created in order to outline the approaches towards environmental, social and economic sustainability, and will be adhered throughout the design process. The environmental sustainability aspect evaluates the level of pollution, usage of scarce materials and energy during manufacturing and operation of the system. Social sustainability evaluates the level at which the system improves the overall quality of life of the community of users and surrounding non-users. Economic sustainability is achieved through a system which maintains a profitable strategy on the long term. 

The main approach adapted for incorporating sustainability in the design is selecting particular criteria designated to sustainability for the trade-off of different UAM concepts. In order to account for environmental sustainability, the trade off will include the measure of energy used per passenger per kilometre. This allows for comparison between the total energy efficiencies of the systems. Social sustainability is evaluated by assessing the noise emissions and down wash produced by the vehicle. These criteria are considered to be driving measures for the level of disturbance in the system on the users and community. Economic sustainability is evaluated through the criteria of development, manufacturing and operation costs, and also includes the average ticket prices per trip.


\section{Trade-off Method}
In the concept selection procedure, several trade-offs will be performed. There is no standard way of approaching such a trade-off, rather there are several different options that can be used. Four different trade-off methods were evaluated: the Experience Based Trade-Off, the Traditional Method, the Analytical Hierarchy Process (AHP) and the Graphical Comparison Method. The Experience Based Trade-Off and Graphical Comparison Method were disregarded because the team does not have the required experience and expertise within this field to base trade-off decisions on. Furthermore, the Traditional Method and AHP have great resemblance, they both assign a weight factor to the trade criteria. The weighting factors are based on the ranking of the criteria, whereas the weighting factors of the AHP are given dependent on the level of importance. Therefore, the AHP is chosen as the trade-off method. 


\section{Trade-Off Criteria and Tools}
In this section first the reasoning behind all the trade-off criteria is shortly discussed and their given weight factor is given. Thereafter it will be explained how the tool that was created was used to evaluate the different concepts. 

\subsection{Trade-Off Criteria}
To determine the trade-off criteria the mission need statement of the project was re-evaluated. This led to the conclusion that the system should provide a (partial) solution to the congestion issue of 2050, while also being sustainable, socially accepted and profitable. These requirements were translated into criteria. Sustainability is checked by comparing the energy demand/pax/km/day (PREE) and the battery mass. Social acceptance was assessed by determining the downwash and noise of each concept. Furthermore, also safety and passenger experience were taken into account. Since these are qualitative criteria, each concept was scored by every team member to determine its average score. The profitability of each concept was determined by finding the ticket price/pax/km. Also the development cost were taken into account, however, since no exact number could be put on this, it is treated a a qualitative criterion at this stage of the project. It is analysed by looking at the Technical Readiness Level (TRL) of the design as a whole. Finally, the technical risks of every concept were evaluated by determining the score of the following qualitative criteria: the ATM/UTM efforts needed, susceptibility to weather conditions and required modifications of current regulations. The weights of all the quantitative and qualitative criteria were determined by use of the AHP. Resulting in \autoref{AHPtableES} shown below.

\begin{table}[H]
\centering
\captionsetup{justification=centering}
\caption{Weightsof the criteria and sub-criteria}
\label{AHPtableES}
\scalebox{0.8}{%
\begin{tabular}{|l|l|c|c|}
\hline
\multicolumn{1}{|c|}{\textbf{Criteria (weight)}}      & \multicolumn{1}{c|}{\textbf{Sub-criteria}} & \textbf{Individual weight}     & \textbf{Total weight}    \\ \hline
\multirow{2}{*}{\textbf{\begin{tabular}[l]{@{}l@{}}Ecological Sustainability\\ (0.36)\end{tabular}}} & Energy demand/hour                & 0.781                 & 0.278                 \\ \cline{2-4} 
                                                                                        & Battery mass  & 0.219                 & 0.078  \\ \hline
\multirow{4}{*}{\textbf{\begin{tabular}[l]{@{}l@{}}Social Acceptance \\ (0.31)\end{tabular}}} & Safety               & 0.593                      & 0.185                      \\ \cline{2-4} 
                                                                                       & Passenger experience & 0.179                      & 0.056                      \\ \cline{2-4} 
                                                                                       & Noise                & 0.176                      & 0.055                      \\ \cline{2-4} 
                                                                                       & Downwash             & 0.053 & 0.017 \\ \hline
\multirow{2}{*}{\textbf{\begin{tabular}[l]{@{}l@{}}Cost/Profit \\ (0.19)\end{tabular}}} & Development cost & 0.191                 & 0.036                 \\ \cline{2-4} 
                             & Ticket price  & 0.809                 & 0.153                               \\ \hline
\multirow{3}{*}{\textbf{\begin{tabular}[l]{@{}l@{}}Technical Risk\\ (0.14)\end{tabular}}} & ATM/UTM efforts                      & 0.569                 & 0.081                 \\ \cline{2-4} 
                                                                                          & Susceptibility to weather conditions & 0.304                 & 0.043                 \\ \cline{2-4} 
                                                                                          & Modification of regulations          & 0.127                 & 0.018                 \\ \hline
\end{tabular}%
}
\end{table}


\subsection{Tool}
In order to get a value for all the quantitative criteria, a set of subtools were created. All these subtools were then integrated to form one tool. The tool is integrated in such a way that any user could use it easily. In addition to this, a database was created by collecting data on existing urban electric and hybrid aircraft currently in development from various sources. This database was used to determine statistics that could be used to estimate the vehicle input parameters of the different concepts. The estimated vehicle parameters form the inputs of the tool. Furthermore, certain parameters were fixed a priori. A sensitivity analysis was performed on the effect of fixing these parameters. The tool asks for certain vehicle and operational parameters after which it provides a set of outputs for both vehicle and infrastructure related parameters. The tool provides a very practical and easy way of optimising the designed concepts. Also, it ensures that all the concepts are evaluated in a similar manner. Finally, the different designs were optimised by making use of the sensitivity plots. These plots showed what the optimal value was for certain parameters in case they were plotted against another. 

\section{Reducing the Design Space}
After evaluating the design options for the vehicle, it was decided that electric jet propulsion is the only viable option for mass expulsion. For rotary wing configurations, all rotors and fans are kept as viable design options. Furthermore, for rotor options only the coaxial configuration is kept. In terms of power source design options, only rechargeable batteries are feasible. A database was created with 26 existing batteries as per May 2019, giving the cell characteristics of each cell. This database is necessary to make an educated evaluation of what battery to choose for a specific vehicle concept.

For the payload configuration it was decided that the cabin shall be closed and that passengers shall be seated in a conventional way. Finally, for the control options, options that are disregarded are rocket thrusters, due to lack of practicality. Other control options omitted are wing deformations due to lack of existing technology. Gimbal control options are also omitted as they are typically too large. 
After analysing the different options for the system, it was found that only the port-to-port system will be considered. This system is more energy efficient and also results in lower ticket prices than all other type of systems. Furthermore, it is assumed that autonomous control is possible in 2050. Therefore, having piloted vehicles is omitted as design option as well.

\section{Trade-Off}
It was decided that for three different capacity of passengers per vehicle ranges concepts would be designed. The capacity ranges are 1-2, 4-6 and 20+ passengers. For each capacity range three concepts were thought of. Every concept was evaluated by making use of the tool. By comparing the outputs of all the concepts, a winning concepts was chosen for each capacity range. The input and output values of each winning concepts are presented in \autoref{inputswinES} and \autoref{outputswinES} respectively. The concepts will be scored on the established quantitative and qualitative criteria on a scale from 1 to 5. Then, the weight factors are added, leading to the final winning concept. 

\begin{table}[H]
\centering
\captionsetup{justification=centering}
\caption{Inputs of winning vehicles}
\label{inputswinES}
\scalebox{0.85}{%
\begin{tabular}{llll}
\hline
\textbf{Parameter}                    & \textbf{1-2 pax}      & \textbf{4-6 pax} & \textbf{20+ pax}    \\ \hline
MTOW {[}kg{]}                         & 575                   &    971              & 7418                \\
OEW/MTOW        & 0.40                    &   0.35          & 0.45                  \\
\# Passengers {[}-{]}                 & 2                     &      4            & 20                  \\
Range {[}km{]}                        & 60                    &       60           & 60                  \\
Max Dimension {[}m{]}                 & 5                     &      10            & 18                  \\
%Battery Energy density {[}Wh/kg{]}    & 260                &      260            & 260               \\
%Battery Power density {[}W/kg{]}      & 2100               &          2100        & 2100              \\
L/D {[}-{]}                           & 12                    &       15           & 13                  \\
Radius of rotors (x number of rotors) & 0.85 (2x) \& 0.8 (2x) &    1 (x4); 1.5 (x1); 0.7 (x2)             & 2.5 (x4), 0.85 (x4) \\
Cruise Velocity {[}m/s{]}             & 32                    &        82          & 111.1               \\ \hline
\end{tabular}
}
\end{table}

\begin{table}[H]
\centering
\captionsetup{justification=centering}
\caption{Outputs of winning vehicles}
\label{outputswinES}
\scalebox{0.85}{%
\begin{tabular}{llll}
\hline
\textbf{Parameter}                          & \textbf{1-2 pax} & \textbf{4-6 pax} & \textbf{20+ pax} \\ \hline
\# Vehicles {[}-{]}                         & 12900             &     3350             & 1500             \\
\# Pads {[}-{]}                             & 1620             &        800          & 550              \\
\# Trips/day {[}-{]}                        & 308000           &       124000           & 26000            \\
\# PREE {[}Wh/(pax-km){]}                   & 790           &        670          & 1230           \\
\# Ticket price per kilometre {[}\$/km{]} & 2.34             &    3.85              & 7.84             \\
\# Ticket price per kilometre {[}\$/km - pad costs{]} & 0.62             &    0.42              & 0.59             \\
\# Passengers/day {[}-{]}                   & 32700           &     269000             & 259000           \\
Vertiport Area {[}m\textsuperscript{2}{]}   & 630            &     2870             & 5700             \\
Total System Area {[}km\textsuperscript{2}  & 1.01          &       1.55           & 3.12          \\
Total Board Time {[}s{]}                    & 210              &        300          & 1020             \\
Noise {[}dBA{]}                             & 71             &       74           & 89             \\ 
Downwash [m/s]                              & 51            &       33              &   48      \\ \hline
\end{tabular}
}
\end{table}



\section{Sensitivity Study}
In such a multidimensional problem with complex dependencies, numerous assumptions are required to simplify the problem and develop useful tools. As such, a sensitivity analysis of the output of the tool was done to look for risk factors, and issues before making the final trade-off.

All parameters of the tool were considered in the analysis, and compared to their effect on the outputs critical to trade-off. While many parameters showed little sensitivity, some showed important relations and limitations of the tool.

The pad costs were found to heavily drive the ticket price, accounting for up to 80\% of the price. These costs may be overestimated or subsidised in other ways. In the tool, the pad costs were linked directly to vehicle size, although this may not be the case.

Limiting the trip range to a minimum required distance improves the energy efficiency and reduces cost. Further cost reduction can be achieved if the market share is increased as well. The market share is powerful enough to override many other sensitivities on the final ticket price.

Different cruise altitudes result in different concept winner between the 2 and 20 passenger vehicle, but not the 4 passenger vehicle. The cruise velocity also impacts ticket price and results in a different winner depending on market share. The 4 passenger concept generally performs better in this criteria. 

Generally speaking, increases to MTOW increase the ticket price locally for each concept. A reduced L/D ratio in cruise also increases the ticket price. Since this is difficult to estimate, may have an important effect on the final trade-off. However, due to the relation between charge rate density and battery mass (and indirectly the energy density), the charging time may decrease as a battery increases in mass. While a heavier vehicle may increase costs, a faster charge time may reduce the final ticket price. This relation is further confounded by the non-linearity of the relation of MTOW to trip energy used. The sensitivity in this relation should be further evaluated in the design phase.


%super short summary 





