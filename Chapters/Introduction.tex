\newpage
\chapter*{\vspace{-10mm}Introduction}
\addcontentsline{toc}{section}{Introduction}

The world has become increasingly urbanized in the past few years. The United Nations predicts that by 2050, 68\% of the world population will live in urban areas\footnote{\url{https://www.un.org/development/desa/en/news/population/2018-revision-of-world-urbanization-prospects.html}[accessed 02.05.19]}. Currently, cities have to cope with pollution, traffic congestion and a continuously decreasing amount of space. The increasing urbanization will amplify these problems in the future. Skyscrapers are a good solution for the shortage of space, as skyscrapers significantly increase the amount of effective square meters in a city.
%A good solution for the shortage of space is the building of skyscrapers, since skyscrapers significantly increase the amount of effective square meters there are per square meter of ground area. 

As of now, the only way to transport people within a city is either over or underneath the ground using existing roads or railways. This limits the amount of people that can be transported over a certain distance at once. Together with the increasing urbanization, these are the main causes of the congestion problems within large cities. An aerial transport system could play an important role in resolving the congestion problem in big cities as it is a three-dimensional transport system. 
%will make the traffic three-dimensional. 

%The congestion problem is not a problem on itself, as it also causes major local air pollution. This is another reason why there is a need for such a new transport system.

Major air pollution is another problem associated with traffic congestion, which leads to another reason for the need of implementing a new transport system. Furthermore, sustainability and environmental impact are becoming more and more important in today's society. Therefore, it is crucial that the new transport system is sustainable as well. So, everything from the operation of the system to the sourcing of raw materials, manufacturing and recycling of any component of the whole system is something that will have to be carefully thought through.

The purpose of this report is to design three different urban air mobility systems. In order to determine the demands for the system, an extensive market analysis into Los Angeles (LA) has been done. This will be used to determine the locations within LA where the system will be implemented leading to a network of various routes. Each system has a different passenger capacity per vehicle, namely 1-2 passengers, 4-6 passengers and a 20 plus vehicle. The designs of these systems will differ as they handle the demand of users in a different way. Several vehicle concepts can operate within each system. To come up with the best vehicle and overall system, a trade-off has been made between different vehicle concepts within each system. Finally, a trade-off will be made to define the "best" system associated with the "best" vehicle for that specific system. 

In \autoref{ch-market}, the Extended Market Analysis is discussed, followed by the Sustainable Development Strategy for this part of the project in \autoref{ch-SDS}. Before the trade-off can be performed, first multiple methods are compared and the best suitable method is chosen in \autoref{ch4-method}. The different criteria and the tools made to analyse them are discussed in \autoref{CriteriaTools}, this is followed by a description of the design space in \autoref{ch-DS}. The preliminary concepts and the trade-off are described in \autoref{ch-tradeoff}, with a sensitivity analysis performed in \autoref{ch:sensitivity}. Finally the revised Project Logic Diagrams are implemented in \autoref{ch-diagrams}. Finally, in \autoref{Conclusion} the conclusion will be presented.